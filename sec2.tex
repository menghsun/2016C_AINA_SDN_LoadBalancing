%\hspace{24pt}
In this section, we introduce SDN architecture and existing load balancing techniques. In order to balance the total AP load, various association control schemes have been proposed in the past.

% ====remove SDN section by little six====
% \subsection{Software Defined Network}
% SDN\cite{mckeown2008openflow} is the key enabler in the new networking architecture innovation, which makes packet forwarding become more flexible and easy-management. It decouples the forwarding and control planes of switches, which means that the sets of forwarding rules in switches can be defined in any way, and the network control is all conducted by SDN controller to have total network view. This revolution allows network administrator to design network experimentation over production and academic network. Figure \ref{fig:SDN-Architecture} shows the architecture of SDN.

%% Fig2.1
% \begin{figure}[tbp]
% \begin{center}
% \includegraphics[width=3.4in]{images/SDN_architecture.pdf}
% \end{center}
% \caption{SDN Architecture}
% \label{fig:SDN-Architecture}
% \end{figure}
%\clearpage

% The Stanford University proposed a wireless SDN platform in Clean Slate plan, called “OpenRoads” \cite{yap2010openroads}. It is built on OpenFlow and SNMP, so it allows researchers to design mobility management mechanism by changing the forwarding rule or controlling the wireless AP configurations. For example in \cite{kim2014seamless} and \cite{dely2013software}, the issues of handover and performance anomaly are discussed based on SDN. The approach of OpenRoads project shows that continued innovation and development in wireless SDN can be expected.



% \subsection{Association Control}
In this section, we introduce some existing user-to-AP association decision schemes which were proposed to balance the total AP load. We classified these schemes into three main categories, namely \emph{strongest signal first}, \emph{least load first} and \emph{Cell Breathing}.

\begin{itemize}
	\item $\emph{Strongest Signal First}$: The Strongest Signal First (SSF) method is the traditional association mechanism in IEEE 802.11 standard. If a user has many APs to choose, it will select the AP with the largest Received Signal Strength Indicator (RSSI) value to associate with. In \cite{teng2009d} and \cite{wu2007proactive}, user are setting to re-associated with the stronger signal strength AP. The main problem of these scheme is that, it ignores the load of AP. If too many users associate to the strongest signal AP at the same time, these users may receive worse performance from this overloading AP.
	\item $\emph{Least Load First}$: Least Load First (LLF) is a widely used load balancing heuristic in which a user selects the least-load AP that he can reach. \cite{papanikos2001study} proposed an association metrics by considering the number of users currently associated with AP. \cite{balachandran2002hot} proposed an association selection scheme that users will associate to the AP which can provide a minimal user-required bandwidth. \cite{bejerano2004fairness} Consider the fairness among all users in the association control.
	\item $\emph{Cell Breathing}$: The cell breathing concept has been studied mostly in CDMA cellular network. In \cite{bahl2007cell} and \cite{bejerano2009cell}, they applied the cell breathing method to IEEE 802.11 WLANs. Figure \ref{fig:cell-breathing} illustrates the example of cell breathing method. First, these APs are associated with 1, 7, 1 users in Figure \ref{fig:fig2_2a}. Through adjusting the power level of APs, some users of AP b shift to adjacent APs, and then the APs are all associated with three users in Figure \ref{fig:fig2_2b}. In Cell Breathing method, an access controller can receive the load of APs, but it has no information of users' positions. The controller can only adjust the beacon power of highest load AP without any prediction of association change trend.
\end{itemize}

%% Fig:2.2
\begin{figure}
	\centering
		\subfigure[All APs have the same power level.]
		{
			\includegraphics[scale=0.3]{images/cb_before.png}
			\label{fig:fig2_2a}
		}
		
		\subfigure[ AP $b$ transmits with the lowest power level.]
		{
			\includegraphics[scale=0.3]{images/cb_after.png}
			\label{fig:fig2_2b}
		}
		
	\caption{Using Cell Breathing Method for Imbalance Situation.}
	\label{fig:cell-breathing}
\end{figure}

In this paper, we propose a load-balancing scheme by combining the cell breathing method with SDN. For the related work of cell breathing method, \cite{bejerano2009cell} proposed an algorithm to determine global optimal solutions for inter-AP fairness. Howerver, both \cite{bahl2007cell} and \cite{bejerano2009cell} cannot adjust the AP power level adaptively due to limited information of AP loads. Through SDN, we can use the global vision to collect the load of AP and users' RSSIs accurately. In this way, our scheme is adaptive. Futhermore, the users' positions can be derived by users' RSSI according to \cite{zaruba2007indoor}. 
% above Dotto

