%\hspace{24pt}
In recent years, the number of intelligent mobile devices has rapidly increased \cite{tai2015comparative}. The penetration rate of wireless Internet has already increased to $91.5\%$ in 2014, which was $14\%$ eight years ago \cite{survey2014comparative}. To meet the growth of user traffic load in Wireless Local Area Network (WLAN), more and more wireless Access Points (APs) are constructed around us. There are currently about 47 million APs worldwide, and the number of APs is expected to be seven times in 2018 \cite{iPassSurvey}.

In some occasions (conference, classroom, etc.), hundreds of wireless devices attempt to associate with the same AP in a short time. The overloading AP infers low throughput problem for users and load imbalanced problem in WLANs. Therefore, an adaptive mobility management is more important. However, there are some difficulties in current wireless architecture. First, due to the distributed management mechanism, it is difficult to add a new feature (e.g. new association control mechanism) into all of the low-level APs. Second, many personnel costs are required to configure the forwarding rules of all the APs and to avoid conflicts among these rules.  Last, the signal of an AP is usually interfered by the other nearby APs owing to the lack of cooperation among APs. All we need is a more resiliently, scalable and centralized-control network architecture.

Software Defined Network (SDN) is a new networking paradigm that provides a more flexible and programmable architecture. SDN separates the control modules from the infrastructure layer to the centralized controllers. OpenFlow \cite{mckeown2008openflow} is the first southbound interface, which defines the protocol between the controllers and switches, in SDN. OpenFlow realizes that the switches from different vendors can cooperate with each other and providing more choices in selecting infrastructure components (e.g. APs, switches). Besides, the SDN allows the network administrators to write programs on the controllers. Hence, we use the programmable architecture to apply our association control mechanism into the wireless network resiliently and easily.

We propose an adaptive load balancing scheme through association control. By the concept of SDN, we use controllers to collect the load of APs and user connection situations. Based on the global vision of all APs, we designed an algorithm for controller to do association control. At the moment when controller detect the load of APs is imbalanced, controller can adaptively configure the beacon power of all APs. Based on the ability to adjust user-to-AP association resiliently, we propose adaptive load balancing scheme to improve the performance in WLAN.

% above Dotto

