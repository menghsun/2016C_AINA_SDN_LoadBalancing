%\hspace{24pt}
In recent years, the number of intelligent mobile devices has rapidly increased \cite{tai2015comparative}. The penetration rate of wireless Internet has already increased to $91.5\%$ in 2014, which was $14\%$ eight years ago \cite{survey2014comparative}. To meet the growth of user traffic load in Wireless Local Area Network (WLAN), more and more wireless Access Points (APs) are constructed around us. There are currently about 47 million APs worldwide, and the number of APs is expected to be seven times in 2018 \cite{iPassSurvey}.

As the number of mobile devices increases, the requirement of high transmission stability and high throughput also increases. Since, mobile devices are usually held by humans, the high mobility of humans may significantly influence the performance of WLAN. Generally, the coverage of Institute of Electrical and Electronics Engineers (IEEE) 802.11 APs is 30-100 meters indoors and 300 meters outdoors. Owing to the high mobility of human, the mobility mechanisms become quite important in current wireless network. When a mobile device moves between APs, it tries to access a target AP which has better signal strength. If there are too much users associate to the same AP, these user will have low bandwidth and low throughput. To meet the load imbalance problem in WLANs, an adaptive mobility management is more important.

IEEE 802.11 is the major standard family of WLAN \cite{ieee2001ieee}. The transmission rate of an AP is up to 54 Mbit/s, both in 2.4 and 5 GHz of Industrial Scientific Medical Band. In response to the popularity of mobile devices and the high popularity of wireless Internet, the wireless Internet throughput has increased a lot. Generally, in some occasions (i.e. conference, classroom), hundreds of wireless devices attempt to associate with APs in a short time. The amount of users an AP can serve is related to the bandwidth of wire backhaul. We need to ensure there are enough APs installed and connected to the wired network to support the client traffic.

However, based on current hierarchical network infrastructure, it is difficult to manage new APs for mostly enterprises and campus. It takes many personnel costs to configure the forwarding rules of all the APs, and to ensure that there is no conflict in these rules. Due to lack of centralized management mechanism in legacy WLAN, it is difficult to add a new feature (e.g. new association control mechanism) into all of the low-level APs. Secondly, the signal of an AP is usually interfered by the other nearby APs and there is no cooperation and no coordination between them. All we need is a more resiliently, scalable and centralized-control network architecture.

Software Defined Network (SDN) is a new networking paradigm that provides a more flexible and programmable architecture. It separates the control modules from the infrastructure layer to the centralized controllers. OpenFlow protocol \cite{mckeown2008openflow} is one of SDN protocols, and it was proposed as the interface between the controllers and switches. OpenFlow realizes that the switches from different vendors can cooperate with each other and providing more choices in selecting infrastructure components (e.g. APs, switches). The SDN allows the network administrators to write programs on the controllers. Thanks to the programmable architecture, we can apply our association control mechanism into the wireless network resiliently and easily.

We propose an adaptive load balancing scheme through association control. By the concept of SDN, we use controllers to collect the load of APs and user connection situations. Based on the global vision of all APs, we designed an algorithm for controller to do association control. At the moment when controller detect the loads of APs are imbalance, controller can adaptively configure the beacon power of all APs. Based on the ability to adjust user-to-AP association resiliently, we propose adaptive load balancing scheme to improve the performance in WLAN.

% above Dotto

